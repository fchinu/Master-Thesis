\begin{abstract}
    \fancyhead[C]{Abstract}

    Quantum Chromo-Dynamics (QCD) is the theory describing the strong interactions between quarks and gluons, the fundamental constituents of hadrons. Lattice QCD calculations predict a transition from colour-neutral hadrons to a colour-deconfined state called Quark-Gluon Plasma (QGP) under extreme temperature and energy density conditions, which can be reached in the laboratory by colliding high-energy heavy ions.
    
    Heavy-flavour quarks (charm and beauty) are produced in the earliest stages of the collision and interact with the formed medium, losing energy through interactions with its constituents, making heavy-flavour hadrons excellent probes of the properties of the QGP. The hadronisation process is the transition from colour-charged partons (quarks and gluons) produced in a collision into colour-neutral hadrons. It is typically parametrised using fragmentation functions, assuming the universality of the process (i.e., independence from the collision system and energy) and the independence of the quark hadronisation via fragmentation. This approach fails in describing the measured charm-baryon production measured in proton-proton (pp) and p--Pb collisions at the Large Hadron Collider (LHC), suggesting that the hadronisation process is not universal. The hadronisation mechanism is expected to be modified by the presence of the QGP, as a novel process, called recombination, is expected to occur. In this process, the produced heavy-flavour quarks combine with other quarks from the medium to form hadrons. The observed baryon enhancement in pp collisions is qualitatively described by models that include QGP droplets formation, as well as by event generators such as \textsc{Pythia}~8 in which colour reconnections are considered in modelling the parton shower and the fragmentation in the parton-rich environment created in pp collisions at the LHC.
    
    In the presence of QGP, the production of strange quarks is expected to be enhanced due to the increase in their thermal production owing to the high temperatures reached in the medium. An increased production of strange hadrons relative to pions is observed in Pb--Pb collisions with respect to pp collisions, where the production of QGP is not expected. However, a smooth increase of strange-hadron relative abundances with the number of charged-particles produced in the collision has been observed in pp collisions, raising the question of whether small droplets of QGP could also be produced in high-multiplicity pp collisions.

    Therefore, because of the recombination mechanism and the enhanced production of strange quarks, an increase in the strange over non-strange \ds/\dpl production yield ratio is expected in the presence of QGP.
    
    The ALICE detector installed at the LHC is designed to address the physics of strongly-interacting matter and QGP produced in ultra-relativistic heavy-ion collisions. During the LHC Long Shutdown 2 (2019--2021) the detector was upgraded by enhancing the tracking performance and increasing the readout rate to collect larger data samples, improving its capabilities to probe the QGP with heavy-flavours.
    
    This Thesis is devoted to the precise measurement of the transverse-momentum- (\pt-)differential \ds/\dpl production-yield ratio at midrapidity ($\lvert y\rvert < 0.5$) in pp collisions at \thirteen with the data collected by the ALICE experiment during the ongoing LHC Run~3 data-taking period. This ratio of strange over non-strange charm meson yields allows for a direct access to information on charm-quark hadronisation mechanisms. Due to their small lifetime ($\tau \sim 100-300~\si{\micro\meter}/c$), \ds and \dpl mesons cannot be directly detected and are reconstructed through their decay products.
    They are reconstructed through the same hadronic decay channel
    \begin{equation*}
        \ds, \dpl \rightarrow \mathrm{\phi\pi^+ \rightarrow K^+K^-\pi^+}\quad ,
    \end{equation*}
    allowing for the cancellation of some of the systematic uncertainties related to the measurement. Multiclass Machine Learning (ML) algorithms have been employed to suppress the large combinatorial background arising from the combination of three independent tracks produced in the pp collision and increase the statistical significance of the measurement. Additionally, the ML-based selections were used to increase the relative contribution of prompt \ds and \dpl mesons (i.e., those directly produced in the hadronisation of a charm quark or through the strong decay of a directly produced excited charm hadron or charmonium state) in the selected sample. The signal is extracted in 14 \pt intervals within the $0.5 < \pt < 24$~\gevc range by fitting the invariant mass distribution of candidates passing the ML selections. 

    The extracted signal is then corrected for the geometrical acceptance of the ALICE detector, the selection efficiency, and the residual non-prompt contamination arising from D mesons produced in the decay of a beauty hadron and surviving the ML selections.
    
    The measured \ds/\dpl production-yield ratio is compared to results obtained by the ALICE Collaboration with the data collected during the LHC Run~2 data-taking period at the different centre-of-mass energies of $\sqs = 5.02, 7$, and 13~\tev and with measurements performed by the LHCb Collaboration at the LHC in the forward-rapidity range $2.0 < y < 4.5$ in pp collisions at $\sqs = 13$~\tev. The results are compatible with those obtained in Run~2 by both ALICE and LHCb Collaborations, indicating no significant dependence of the \ds/\dpl ratio on the centre-of-mass energy and rapidity. Thanks to the larger data samples collected during the LHC Run~3 data-taking period, a more precise and granular measurement of the \ds/\dpl production-yield ratio than that measured in Run~2 is achieved. Furthermore, the \pt reach of the measurement has been extended to lower values, reaching as low as 0.5~\gevc. These measurements provide state-of-the-art results on the production of charm-strange mesons in pp collisions.
    
    Lastly, to study the performance of the ALICE experiment in heavy-ion collisions, the reconstruction of \ds and \dpl mesons was performed in Pb--Pb collisions at \fivenn for different centrality intervals, which represent the degree of overlap of the two colliding nuclei, using rectangular selection criteria. The results in the 10--30\% centrality class have been used as benchmarks of the performance of the upgraded ALICE experiment in heavy-ion collisions. These results will provide a solid baseline for the study of the \ds/\dpl production-yield ratio in Pb--Pb collisions, to be performed in the future. By comparing the results obtained in Pb--Pb collisions and the measurements in pp collisions described in this Thesis, insights into the hadronisation mechanisms of charm quarks in the presence of the QGP, where strange quarks are more abundant, will be obtained.
\end{abstract}