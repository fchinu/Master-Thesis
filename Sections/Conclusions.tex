\chapter{Conclusions and perspectives}\label{ch:conclusions}

This Thesis presented the measurement of the \pt-differential strange over non-strange \ds/\dpl production-yield ratio in proton-proton collisions at \thirteen, performed with the upgraded ALICE detector and the data collected during the LHC Run~3 data-taking period. 

The analysis was performed via the full reconstruction of displaced decay topologies through the same hadronic decay channel $\ds, \dpl \rightarrow \phi\pi^+\rightarrow\mathrm{K^+K^-\pi^+}$. Machine Learning models using the XGBoost algorithm were employed to enhance the selection efficiency of the signal candidates and to reduce the combinatorial background. The measurement was carried out in the transverse-momentum range $0.5<\pt<24$~\gevc, and extended the \pt coverage at low \pt with respect to previous measurements performed by the ALICE Collaboration at $\sqs = 5.02, 7$ and $13$~\tev, reported in Refs.~\cite{ALICE:2021mgk,ALICE:2017olh,ALICE:2023sgl}, respectively. Additionally, thanks to the larger data sample collected during the LHC Run~3 data-taking period, and the reconstruction of both D-meson species in the same decay channel, the results presented in this Thesis significantly reduced both the statistical and systematic uncertainties of the measurement, and improved its granularity with respect to the measurements performed by both the ALICE and LHCb~\cite{LHCb:2015swx} Collaborations at mid and forward rapidities, respectively.

This measurement provides state-of-the-art results for the understanding of the hadronisation mechanisms (i.e., the transition from colour-charged partons (quarks and gluons) produced in a collision into colour-neutral hadrons) of charm quarks in high-energy hadronic collisions. As described in Chapter~\ref{ch:openHF}, the hadronisation mechanism is expected to be modified by the presence of a deconfined medium, the Quark-Gluon Plasma (QGP), formed in high-energy nuclear collisions, as hadrons can be formed through coalescence of charm quarks, which are produced before the QGP is formed, with light quarks from the medium. Additionally, in the presence of a QGP, the production of strange quarks is expected to be enhanced, as the high temperatures reached in the medium allow for the thermal production of strange-antistrange quark pairs. The measurement of the \ds/\dpl production-yield ratio is a powerful tool to investigate the hadronisation mechanisms of charm quarks, and is a sensitive probe to the phenomenon of strangeness enhancement. 

Several measurements confirmed the production of the QGP in high-energy nuclear collisions, such as Refs.~\cite{NA50:1997hlx,WA97:2000apo,ALICE:2018vuu}. The production of an extended QGP phase is not expected in proton-proton collisions, as the system size is too small to reach the critical energy density required for the QGP formation. However, recent measurements performed at the LHC~\cite{CMS:2016fnw,CMS:2010ifv,ALICE:2019zfl} provide evidence of the presence of collective effects in small systems, such as proton-proton and proton-lead collisions, which are usually associated with the formation of a QGP. The ALICE Collaboration reported the observation of a smooth increase in the strange-hadron production with the charged-particle multiplicity in proton-proton collisions at $\sqs=7$~\tev~\cite{ALICE:2016fzo}, reaching, for the highest multiplicity classes, values compatible with those measured in lead-lead collisions. 

\begin{figure}[tb]
    \centering
    \includegraphics[width=0.7\textwidth]{Figures/Chapter 9/DsD0Ratios_LowHighMult_V0M_Derived.pdf}
    \caption{Strange over non-strange \ds/\dz production-yield ratio as a function of \pt for two different multiplicity classes measured at midrapidity ($\lvert y\rvert<0.5$) in proton-proton collisions at $\sqs=13$~\tev by the ALICE Collaboration~\cite{ALICE:2021npz}. Figure taken from the ALICE figure repository~\cite{ALICE_figures}.}
    \label{fig:ALICE_DsD0VsMultiplicity}
\end{figure}

Previous measurements of the multiplicity-dependence of the strange over non-strange \ds/\dz production-yield ratio performed at midrapidity ($\lvert y\rvert<0.5$) by the ALICE Collaboration~\cite{ALICE:2021npz} do not provide a clear indication of the strangeness enhancement in proton-proton collisions at $\sqs=13$~\tev. The results are illustrated in Fig.~\ref{fig:ALICE_DsD0VsMultiplicity}, where the measured \ds/\dz production-yield ratio is shown as a function of \pt for two different multiplicity classes. The results show a slight increase of the \ds/\dz production-yield ratio with the charged-particle multiplicity, although the two \pt-differential measurements are compatible within their uncertainties. The results presented in this Thesis provide a solid foundation for future studies of the strangeness enhancement in the heavy-flavour sector. The strategy of using the \dpl meson as a reference for the non-strange production, as well as the reconstruction of both \ds and \dpl mesons in the same hadronic decay channel allows for a significant reduction of the systematic uncertainties of the measurement. The measurement of the multiplicity dependence of the \ds/\dpl production-yield ratio may provide more precise results than those achieved through the \ds/\dz production-yield ratio, and, thereby, provide a more sensitive probe to the strangeness enhancement in proton-proton collisions. Additionally, the much larger data sample collected during the LHC Run~3 data-taking period will allow for the extension of the measurement to lower \pt values, where the effects of the strangeness enhancement are expected to be more pronounced, increase the statistical precision of the results and perform a more \pt- and multiplicity-differential measurement.

\begin{figure}[tb]
    \centering
    \includegraphics[width=\textwidth]{Figures/Chapter 9/PromptDs_vs_PromptD_Raa_2pads_1.pdf}
    \caption{\raa of prompt \ds meson and average \raa of prompt \dz, \dpl, and $\mathrm{D^{*+}}$ mesons as a function of \pt for the 0--10\% (left panel) and 30--50\% (right panel) centrality classes measured at midrapidity ($\lvert y\rvert<0.5$) in Pb--Pb collisions at \mbox{$\snn=5.02$~\tev} by the ALICE Collaboration~\cite{ALICE:2021npz}.}
    \label{fig:RAA_Ds}
\end{figure}

\begin{sloppypar}
The multiplicity phase space can be further explored by measuring the strange over non-strange D meson production-yield ratio in Pb--Pb collision, where much higher charged-particle multiplicities are reached. Previous measurements performed by the ALICE experiment at a centre-of-mass energy per nucleon pair of \mbox{$\snn=5.02$~\tev~\cite{ALICE:2021kfc}} did not provide a clear indication for the phenomenon of strangeness enhancement. Figure~\ref{fig:RAA_Ds} shows the nuclear modification factor \raa of prompt \ds mesons, compared to the average \raa of prompt \dz, \dpl, and $\mathrm{D^{*+}}$ mesons, as a function of \pt. Results from the 0--10\% and 30--50\% centrality classes are shown in the left and right panels, respectively. The measured \raa of prompt \ds mesons in Pb--Pb collisions is compatible within uncertainties with that of other non-strange D mesons for $\pt\gtrsim10$\gevc, where the hadronisation process is expected to occur mainly via fragmentation. At lower \pt, where the coalescence mechanism is expected to play a more relevant role, the measured \raa of prompt \ds mesons is systematically higher than that of non-strange D mesons, although the results are compatible within one standard deviation for both central and semicentral collisions.
\end{sloppypar}

\begin{figure}[htb]
    \centering
    \includegraphics[width=\textwidth]{Figures/Chapter 9/Ratio_and_DoubleRatio_DsOverD0_PbPb_pp_5TeV_vs_models_1.pdf}
    \caption{Top panels: \ds/\dz \pt-differential production-yield ratios in the 0--10\% (left panel) and 30--50\% (middle panel) centrality intervals measured in Pb--Pb collisions at \mbox{$\snn=5.02$~\tev} by the ALICE Collaboration and in pp collisions (right panel) at the same centre-of-mass energy measured by the ALICE Collaboration, compared with theoretical calculations based on charm-quark transport in a hydrodynamically expanding QGP and statistical hadronisation. Bottom panels: \ds/\dz \pt-differential ratios in Pb--Pb collisions divided by those in pp collisions, in the 0--10\% (left panel) and 30--50\% (right panel) centrality intervals, compared with theoretical calculations. Figure taken from Ref.~\cite{ALICE:2021npz}.}
    \label{fig:Double_ratio}
\end{figure}

Further insights on the enhancement of strangeness production in Pb--Pb collisions compared to pp collisions can be obtained from the study of the double ratio of strange over non-strange D meson production-yield ratios in the two collision systems. Figure~\ref{fig:Double_ratio} shows, in the top row, the \pt-differential \ds/\dz production-yield ratios in the 0--10\% and 30--50\% centrality classes measured in Pb--Pb collisions at \mbox{$\snn=5.02$~\tev} by the ALICE Collaboration, and in pp collisions at the same centre-of-mass energy. The results are compared with theoretical calculations based on charm-quark transport in a hydrodynamically expanding QGP and statistical hadronisation. The Catania model~\cite{Plumari:2017ntm,Scardina:2017ipo}, already introduced in Chapter~\ref{ch:openHF} for pp collisions, implements the coalescence mechanism in Pb--Pb collisions through the Wigner formalism~\cite{Dover:1991zn} at the phase boundary, and describes the \ds/\dz production-yield ratio in pp and central Pb--Pb collisions. In the TAMU~\cite{He:2014cla} model, a combined recombination and fragmentation approach is implemented. The former is realized via the Resonance Recombination Model (RRM)~\cite{Ravagli:2007xx} where the recombination probability for the two-body case is controlled by resonance amplitudes and is expressed as a relativistic Breit-Wigner cross-section. It significantly overestimates the measured \ds/\dz ratio by a similar amount in the two colliding systems. In the Parton-Hadron-String Dynamics (PHSD) model~\cite{Song:2015sfa} a kinetic approach is applied, and the hadronisation in heavy-ion collisions is described via a Monte Carlo simulation of the coalescence process in competition to fragmentation. It describes the \ds/\dz production-yield ratio in central and semicentral Pb--Pb collisions and pp collisions within uncertainties. Lastly, the GSI-Heidelberg Statistical Hadronisation Model~\cite{Andronic:2021erx} (SHMc), is reported for central and semicentral Pb--Pb collisions. The \pt spectra of charm hadrons are modelled with a core-corona approach. In the low-\pt region, the charm production is dominated by the core contribution, described with a Blast Wave function. The corona contribution is parametrised from measurements in pp collisions and is relevant at high \pt. The \pt-spectra modification due to resonance decays is computed using the FastReso package~\cite{Mazeliauskas:2018irt}. It provides a similar \pt shape for the \ds/\dz production-yield ratio as that provided by the TAMU model.


In the bottom row, the ratio between the \ds/\dz production-yield ratios in Pb--Pb collisions and those in pp collisions is shown for the 0--10\% and 30--50\% centrality classes. The average values of the double ratio in the $2 < \pt < 8$~\gevc interval are larger than unity by about $2.3\sigma$ and $2.4\sigma$ of the combined statistical and systematic uncertainties, for the 0--10\% and 30--50\% centrality intervals, respectively. It is compared with predictions from the TAMU and Langevin-transport with Gluon Radiation (LGR) models~\cite{Li:2019lex} (which implements the coalescence mechanism at the phase boundary), which predict a peak at $\pt\sim3-4$~\gevc, which could be attributed to the different masses of the \ds and \dz mesons and the collective radial expansion of the system, leading to an equal velocity boost to all particles. A similar \pt shape is predicted by the GSI-Heidelberg SHMc model. The Catania and PHSD model predict an almost flat \pt dependence of the double ratio, with a mild decreasing trend with \pt from the latter. 


The presented results do not allow for drawing firm conclusions on the phenomenon of strangeness enhancement in Pb--Pb collisions given the large uncertainties. The double ratio of \ds/\dpl production-yield ratios in Pb--Pb and pp collisions, will provide a clearer picture of the phenomenon. It would doubly benefit from the reduction of the systematic uncertainties (from the reconstruction of both D mesons in the same decay channel), as the improvement would affect both the numerator and the denominator of the double ratio. Additionally, the statistical uncertainties of the measurement will be significantly reduced thanks to the larger data sample collected during the LHC Run~3 data-taking period. This will allow the extension of the measurement to lower \pt values, where the effects of the strangeness enhancement are expected to be more pronounced, due to the onset of the coalescence hadronisation mechanism, and to perform the measurement from the most central to the most peripheral collisions, where the effects of the QGP are expected to be less pronounced. With these perspectives, the ALICE Collaboration will provide a state-of-the-art measurement of the strangeness enhancement in the heavy-flavour sector, with a comprehensive study of the hadronisation mechanisms of charm quarks in high-energy nuclear collisions, from the lowest multiplicities reached in pp collisions to the highest multiplicities reached in Pb--Pb collisions.



