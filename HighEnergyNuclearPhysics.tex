\chapter{High-energy nuclear physics}
\section{Quantum Chromodynamics}
In the mid-20$^{\mathrm{th}}$ century, the realm of particle physics underwent a transformative phase, marked by the discovery of a seemingly endless variety of subatomic particles. This era witnessed the unveiling of numerous mesons and baryons, which left physicists with the necessity of developing a framework that could describe the behaviour of these particles and their interactions. This led to the development of the static quark model, which emerged in the 1960s as a groundbreaking conceptual framework to categorize the various observed particles. Developed independently by Murray Gell-Mann\cite{Gell-Mann:1964ewy} and George Zweig\cite{Zweig:1964jf, Fritzsch:1972jv}, this model postulated the existence of fundamental constituents called quarks, which, in order to reflect the experimental findings, had to be fermions (to describe baryons with spin 1/2 and 3/2) with fractional electric charge. The quark model beautifully explained the organization of hadrons in terms of three quarks ($u$, $d$, and $s$), leading to the development of a more structured and coherent classification of particles.

Despite the phenomenological success of the static quark model, it had two problems: it introduced particles with fractional charge, which had never been observed before, and, most importantly, it gave rise to a violation of the Fermi-Dirac statistics. The $\Delta^{++}$, $\Delta^{-}$, and $\Omega^{-}$ baryons, in fact, have symmetric orbital, spin and flavour wavefunctions, which defied the Pauli exclusion principle that should have implied antisymmetric wavefunctions for these particles.

To resolve these inconsistencies, a new degree of freedom, the \emph{colour}, was introduced. Hadrons wavefunctions were assumed to be totally antisymmetric in colour quantum numbers, effectively implementing the Pauli exclusion principle.

The simplest model of colour would be to assign quarks to the fundamental representation of a global $SU(3)$ symmetry. Each quark now carries a colour index: $q_i$, where $i = 1, 2, 3$, and transforms under the fundamental ($3$) representation of $SU(3)$, while antiquarks,  $\bar{q}_i$, transform in the $\bar{3}$ representation. Introducing the totally antisymmetric tensor $\varepsilon^{ijk}$, possible compositions of quarks that give rise to colour singlets are 
\begin{equation*}
    \bar{q}^iq_i,\qquad \varepsilon^{ijk}q_iq_jq_k,\qquad \varepsilon^{ijk}\bar{q_i}\bar{q_j}\bar{q_k},
\end{equation*}
which are the quarks compositions of mesons, baryons, and antibaryons, respectively. 

One of the tests supporting the existence of colour and fractional electric charge came in the form of the ratio R, of the $\mathrm{e}^+ \mathrm{e}^-$ total hadronic cross-section to the cross-section of a pair of muons produced from the same annihilation process. The virtual photon emitted in the annihilation can produce all electrically charged pairs of particles and antiparticles, as shown in Fig.~\ref{fig:ee_to_ff_diagram}.

\begin{figure}[h]
    \centering
        \feynmandiagram [horizontal=a to b] {
          i1 [particle=\(e^{-}\)] -- [fermion] a -- [fermion] i2 [particle=\(e^{+}\)],
          a -- [photon, edge label=\(\gamma^*\)] b,
          f1 [particle=\(\bar{f}\)] -- [fermion] b -- [fermion] f2 [particle=\(f\)],
        };
\caption{$\mathrm{e}^+ \mathrm{e}^-$ annihilation to a pair of fermions}
    \label{fig:ee_to_ff_diagram}
\end{figure}


The ratio R is given by:
\begin{equation*}
    R = \frac{\sigma(e^+e^- \rightarrow hadrons)}{\sigma(e^+e^- \rightarrow \mu^+\mu^-)} = N_c \sum_f Q_f^2\ ,
\end{equation*}
where $N_c$ represents the number of existing colours and $Q_f$ is the electric charge of the quark flavour $f$. Notably, this ratio is dependent on the energy of the center-of-mass system and encompasses all possible quark flavors that can be produced by the virtual photon at that specific energy level. The experimental data for $R$ (shown in Fig.~\ref{fig:R_vs_s}) exhibited a remarkable agreement with the predictions of the three-color model, thereby providing compelling evidence for the existence of color and fractional electric charge of quarks.

The final step that propelled the development of QCD as a comprehensive theory of the strong force was the insight into the mechanism that ensured all hadron wavefunctions to be color singlets. This emerged from the discovery of asymptotic freedom, a phenomenon observed in deep-inelastic scattering experiments. Non-Abelian gauge theories, often referred to as Yang-Mills theories, were identified as having this unique characteristic. This realization led to the formulation of QCD by elevating the global color $SU(3)$ symmetry to a local one, allowing the 8 quanta of the $SU(3)$ gauge field, called \emph{gluons}, to mediate the strong force, successfully describing the confinement and behavior of quarks and gluons within hadrons.

\begin{figure}[p]
    \centering
    \includegraphics[width=\linewidth]{Figures/Chapter 1/rpp2022-R_udscb.pdf}
    \caption{$R$ as a function of $\sqrt{s}$ in the light-flavor, charm, and beauty threshold regions taken from \cite{pdg}. The green curve is a naive quark-parton model prediction, while the red one is a 3-loops pQCD prediction. Breit-Wigner parameterizations of $J/\psi$, $\psi$(2S), and $\Upsilon$(nS), n = 1,2,3,4 are also shown}
    \label{fig:R_vs_s}
\end{figure}

The QCD Lagrangian density can be written as:
\begin{equation}\label{eq:Lqcd}
    \mathcal{L}_{QCD}=-\frac{1}{4} F^a_{\mu\nu}F_a^{\mu\nu} + \sum_f \bar{q}_f^i (i\gamma^\mu(\mathcal{D}_\mu)_{ij}-m_f\delta_{ij})q_f^j\ ,
\end{equation}
where $F^a_{\mu\nu}$ is the field strength tensor defined in terms of the gluon field $A^a_\mu$ and the $SU(3)$ structure constant $f^{abc}$:
\begin{equation} \label{eq:F}
    F^a_{\mu\nu} = \partial_\mu A^a_\nu - \partial_\nu A^a_\mu + g_s f^{abc}A^b_\mu A^c_\nu 
\end{equation}
and $(\mathcal{D}_\mu)_{ij}$ is the covariant derivative:
\begin{equation*}
    (\mathcal{D}_\mu)_{ij} = \partial_\mu \delta_{ij} - ig_s(t^a)_{ij}A_\mu^a\ ,
\end{equation*}
with $t^a$ being one of the generators of the $SU(3)$ representation.

The last term in Eq.~\ref{eq:F} is peculiar to non-Abelian theories, and gives rise to triplet and quartic gluon self-interactions illustrated in Fig.~\ref{fig:Feynman-gluons}. $g_s$ is the coupling constant, which determines the strength of the interaction between the coloured particles.

\begin{figure}[htb]
    \centering
    \begin{tikzpicture}
      \begin{feynman}
        \vertex (a) {a};
        \vertex [right=of a] (b);
        \vertex[above right= of b] (c) {b};
        \vertex[below right= of b] (d) {c};
        \diagram* {
          (a) -- [gluon] (b),
          (b) -- [gluon] (c),
          (b) -- [gluon] (d),
        };
      \end{feynman}
    \end{tikzpicture} \qquad
    \begin{tikzpicture}
      \begin{feynman}
        \vertex (a);
        \vertex[above left=of a] (b) {a};
        \vertex[above right= of a] (c) {b};
        \vertex[below right= of a] (d) {c};
        \vertex[below left= of a] (e) {d};
        \diagram* {
          (a) -- [gluon] (b),
          (a) -- [gluon] (c),
          (a) -- [gluon] (d),
          (a) -- [gluon] (e),
        };
      \end{feynman}
    \end{tikzpicture}
    \caption{Feynman diagrams for gluons self-interactions}
    \label{fig:Feynman-gluons}
\end{figure}

The second term of Eq.~\ref{eq:Lqcd} describes the interactions between quarks and gluons, sketched in Fig.~\ref{fig:Feynman_q_g}, and contains the mass term for the fermions. It is noteworthy to observe that the interaction between quarks and gluons is diagonal in flavor, meaning that the strong interaction conserves the flavor of quarks. In contrast, colour mixing is allowed within the framework of QCD.

\begin{figure}[htb]
    \centering
    \begin{tikzpicture}
      \begin{feynman}
        \vertex (a);
        \vertex [below=0.1em of a] {$t^a_{ij}$};
        \vertex[left=of a] (b) {$q_i$};
        \vertex[right= of a] (c) {$q_j$};
        \vertex[above right= of a] (d) {a};
        \diagram* {
          (b) -- [fermion] (a),
          (a) -- [fermion] (c),
          (a) -- [gluon] (d),
        };
      \end{feynman}
    \end{tikzpicture}
    \caption{Feynman diagram for quark-gluon interaction}
    \label{fig:Feynman_q_g}
\end{figure}

\subsection{Running coupling constant}
If one considers a dimensionless physical observable, denoted in the following as $R$, which solely depends on a single energy scale, $Q$, one might naturally expect that $R$ would maintain a constant value, independent of the specific energy scale chosen. However, this does not hold true when loop diagrams are studied: the necessity of renormalisation introduces a new energy scale denoted as $\mu$. This scale, known as the renormalisation scale, is the point at which the subtraction of the ultraviolet divergences is carried out. Critically, $\mu$ is an arbitrary parameter and, as such, is non-physical. Consequently, $R$ becomes dependent on the ratio $Q^2/\mu^2$ and the renormalised coupling $\alpha_s = g_s^2/4\pi$: $R = R\left(\frac{Q^2}{\mu^2},\als\right)$. The $\mu$ independence of $R$ (which is an essential requirement given $\mu$'s arbitrariness) can be expressed as:
\begin{equation}\label{eq:RGE}
    \mu^2 \frac{\de R\left(\frac{Q^2}{\mu^2},\alpha_s\right)}{\de \mu^2} = \mu^2 \left[\frac{\partial}{\partial\mu^2}+\frac{\partial \alpha_s}{\partial\mu^2}\frac{\partial}{\partial\alpha_s}\right]R\left(\frac{Q^2}{\mu^2},\alpha_s\right) = 0\ , 
\end{equation}
a fundamental equation known as the renormalisation group equation. This equation is exactly true in the case of a prediction that considers all perturbative orders. If one limits the expansion at a fixed order $\alpha_s^N$, then a dependence of $R$ from $\mu$ is observed at the $\als^{N+1}$ order.\\ Solving Eq.~\ref{eq:RGE} requires the introduction of the concept of the running coupling $\alpha_s(Q^2)$, which evolves as a function of $Q$. By introducing
\begin{equation*}
    t\equiv \mathrm{log}(Q^2/\mu^2), \qquad \beta(\als)\equiv \mu^2 \frac{\de\als}{\mathrm{\mu^2}}\quad ,
\end{equation*}
Eq.~\ref{eq:RGE} can be written as
\begin{equation*}
    \left(-\frac{\partial}{\partial t} + \beta(\als)\frac{\partial}{\partial \als}\right) R(e^t,\als) = 0
\end{equation*}
This first-order partial differential equation can be solved by defining a new function: the running coupling $\als(Q^2)$
\begin{equation}\label{eq:t_integral}
    t = \mathrm{log}(Q^2/\mu^2) \equiv \int_{\als}^{\als(Q^2)} \frac{\de x}{\beta(x)} , \quad \mathrm{with}~\als=\als(\mu^2)\quad .
\end{equation}
By differentiating Eq.~\ref{eq:t_integral} with respect to $t$ and \als, one gets:
\begin{equation}\label{eq:beta_def}
    \beta(\als(Q^2)) = \frac{\partial\als(Q^2)}{\partial t}, \quad \frac{\de\als(Q^2)}{\de\als} = \frac{\beta(\als (Q^2))}{\beta(\als)}\quad .
\end{equation}
It results from this last set of equations that $R(1,\als(Q^2))$ satisfies Eq.~\ref{eq:RGE}; hence, the running coupling constant has absorbed the $\mu$ scale dependence of $R$. As a consequence, the knowledge of $R(1,\als)$, which can be evaluated in fixed-order perturbation theory, allows to know the dependence of $R$ from $Q^2$, which is the physical scale at which the coupling is gauged, by simply substituting $\als \rightarrow \als(Q^2)$. 

\subsubsection{The \ensuremath{\beta} function}
The running of the coupling constant is determined by the $\beta(\als)$ function, which is evaluated from loop corrections to the bare vertices of the theory. As of the time of the writing of this Thesis, the $\beta$ function has been evaluated up to 5 loops\cite{Herzog:2017ohr}. In Fig.~\ref{fig:beta_loops}, the 1-loop Feynman diagrams contributing to the $\beta$ function evaluation are reported.

\begin{figure}[htb]
    \centering
    \begin{tikzpicture}
      \begin{feynman}
        \vertex (a);
        \vertex [right=1cm of a] (b);
        \vertex[right=1cm of b] (c);
        \vertex[right=1cm of c] (d);
        \diagram* {
            (a) -- [gluon] (b)
            -- [fermion, half left, looseness=1.5] (c)
            -- [fermion, half left, looseness=1.5] (b),
            (c) -- [gluon] (d),
        };
      \end{feynman}
    \end{tikzpicture}\quad
    \begin{tikzpicture}
      \begin{feynman}
        \vertex (a);
        \vertex [right=1cm of a] (b);
        \vertex[right=1cm of b] (c);
        \vertex[right=1cm of c] (d);
        \diagram* {
            (a) -- [gluon] (b)
            -- [gluon, half left, looseness=1.5] (c)
            -- [gluon, half left, looseness=1.5] (b),
            (c) -- [gluon] (d),
        };
      \end{feynman}
    \end{tikzpicture}\quad
    \begin{tikzpicture}
      \begin{feynman}
        \vertex (a);
        \vertex [right=of a] (b);
        \vertex[above=1cm of b] (c);
        \vertex[right=of b] (d);
        \diagram* {
            (a) -- [gluon] (b)
            -- [gluon, half left, looseness=1.5] (c)
            -- [gluon, half left, looseness=1.5] (b),
            (b) -- [gluon] (d),
        };
      \end{feynman}
    \end{tikzpicture}\quad

    \vspace{0.6cm}
    \begin{tikzpicture}
      \begin{feynman}
        \vertex (a);
        \vertex [right=1cm of a] (b);
        \vertex[right=1cm of b] (c);
        \vertex[right=1cm of c] (d);
        \diagram* {
            (a) -- [gluon] (b)
            -- [ghost, half left, looseness=1.5] (c)
            -- [ghost, half left, looseness=1.5] (b),
            (c) -- [gluon] (d),
        };
      \end{feynman}
    \end{tikzpicture}\quad
    \begin{tikzpicture}
      \begin{feynman}
        \vertex (a);
        \vertex [right=1cm of a] (b);
        \vertex[right=1cm of b] (c);
        \vertex[right=1cm of c] (d);
        \vertex[below=0.31cm of c] (f) {$ $};
        \diagram* {
            (a) -- [fermion] (b) -- [fermion] (c) -- [fermion] (d),
            (b) -- [gluon, half left, looseness=1.5] (c)
            
        };
      \end{feynman}
    \end{tikzpicture}\quad
    
    \vspace{0.5cm}
    \begin{tikzpicture}
      \begin{feynman}
        \vertex (a);
        \vertex [right=1cm of a] (b);
        \vertex[above right=1cm of b] (c);
        \vertex[right=1cm of c] (d);
        \vertex[below right=1cm of b] (e);
        \vertex[right=1cm of e] (f);
        \diagram* {
            (a) -- [gluon] (b) -- [fermion] (c) -- [fermion] (d),
            (b) -- [fermion] (e) -- [fermion] (f),
            (c) -- [gluon] (e)
        };
      \end{feynman}
    \end{tikzpicture}\quad
    \begin{tikzpicture}
      \begin{feynman}
        \vertex (a);
        \vertex [right=1cm of a] (b);
        \vertex[above right=1cm of b] (c);
        \vertex[right=1cm of c] (d);
        \vertex[below right=1cm of b] (e);
        \vertex[right=1cm of e] (f);
        \diagram* {
            (a) -- [gluon] (b) -- [gluon] (c),
            (b) -- [gluon] (e),
            (f) -- [fermion] (e) -- [fermion] (c) -- [fermion] (d)
        };
      \end{feynman}
    \end{tikzpicture}\quad
    \caption{1-loop Feynman diagrams contributing to the $\beta$ function evaluation}
    \label{fig:beta_loops}
\end{figure}

By limiting the calculations at the first order in the perturbative expansion, one gets:
\begin{equation}\label{eq:beta0}
    \beta(\als) = -\als^2 \frac{11 \mathrm{N_c} - 2 \mathrm{N_f}}{12\pi} + \mathcal{O}(\als^3) \equiv -\als^2 \beta_0 + \mathcal{O}(\als^3)\quad ,
\end{equation}
where $\mathrm{N_c}$ is the number of colours (3), while $\mathrm{N_f}$ is the number of quark flavours which can be considered massless at the physical scale $Q^2$ at which the coupling is being measured.
From Eqs.~\ref{eq:beta0} and \ref{eq:beta_def}, one can extract the $Q^2$ dependency of the running coupling constant:
\begin{equation}\label{eq:alpha_s_running}
    \alpha_s(Q^2) = \frac{\alpha_s(\mu^2)}{1+\alpha_s(\mu^2)\beta_0 \mathrm{log}(Q^2/\mu^2)}\ ,
\end{equation}
Notably, since $\beta_0$ is positive also when considering 6 quark flavours, the strong coupling constant exhibits a monotonic decreasing trend as a function of $Q^2$. This behaviour differs from the one of the electromagnetic coupling constant, which increases with the energy scale due to the screening effect of vacuum polarisation. For QCD, the running of the coupling constant is a direct consequence of the non-Abelian nature of the theory, allowing for gluon self-interactions, which give rise to an anti-screening effect. The idea is that the emission of virtual gluons by static colour sources causes their colour charges to 'leak out' into the surrounding vacuum. Since the interaction between distributions of charges is weaker than the one between point-like charges when the distributions overlap, the effective coupling constant decreases at short distances. This behaviour is known as asymptotic freedom, a key feature of QCD that allows for the perturbative expansion of the theory at high energy scales, where the strong coupling constant is small. At the same time, the running of the coupling constant implies that the theory is non-perturbative at low energy scales, and phenomenological models are required to describe the strong interaction in this regime. Instead of using the renormalisation scale $\mu$ as a free parameter, one can use the running coupling constant to define a physical scale, $\Lambda_{QCD}$, which is the energy scale at which the coupling constant would diverge, if extrapolated outside the perturbative regime. Using Eq.~\ref{eq:alpha_s_running}, one can write:
\begin{equation*}
    \alpha_s(\Lambda_{QCD}) = \frac{1}{\beta_0 \mathrm{log}(Q^2/\Lambda_{QCD}^2)}\quad .
\end{equation*}
The value of $\Lambda_{QCD}$ is determined by its specific definition. However, to obtain the value of the coupling constant measured at $Q^2 = M_Z^2$, an approximate value of $\Lambda_{QCD}$ is around 200 MeV.

Measurements of the running of the coupling constant at different values of $Q$ are illustrated in Fig.~\ref{fig:alpha_s_running} and compared to the theoretical prediction at 5 loops. The agreement between the experimental data and the theoretical prediction is remarkable, confirming the validity of the QCD framework at high energy scales.


\begin{figure}[htb]
    \centering
    \includegraphics[width=0.7\linewidth]{Figures/Chapter 1/Alpha_s_running.png}
    \caption{Summary of measurements of \als as a function of the energy scale $Q$, compared to the running of the coupling computed at five loops, taking as an input the current PDG average, $\als(M_Z^2) = 0.1180 \pm 0.0009$ \gevcc.}
    \label{fig:alpha_s_running}
\end{figure}

\section{(De-)confinement}
The concept of confinement is one of the most intriguing aspects of QCD. It is the phenomenon by which quarks and gluons are never observed as free particles, but are always confined within colour-neutral hadrons. The confinement of quarks and gluons is a direct consequence of the non-Abelian nature of the theory, which, as described in the previous Section, is characterised by an increase of the strong coupling constant at low energy scales. This leads to the formation of colour-neutral hadrons, which are the only particles that can be observed in nature. The confinement of quarks and gluons is a non-perturbative effect, and the theoretical description of this phenomenon is still an open question in QCD. Some phenomenological models, such as the MIT bag model, have been proposed to describe confinement, but a complete understanding of this phenomenon is still lacking. Lattice QCD simulations are the most successful approach to study the non-perturbative regime of the theory, and they have provided a wealth of information on the properties of hadrons and the strong interaction at low energy scales.

\subsection{MIT bag model}
The MIT bag model~\cite{Johnson:1975zp} is a phenomenological model of confinement, which describes hadrons as bound states of quarks and gluons confined within a finite volume, called the bag. The model was developed in the 1970s by A. Chodos, R. L. Jaffe, K. Johnson, C. B. Thorn, and V. F. Weisskopf, and it has been widely used to study the properties of hadrons and the strong interaction. In the MIT bag model, N masslesss fermions are confined within a spherical cavity of radius $R$, which is the bag radius. The confinement arises from a balance between pressure due to the kinematic energy of the fermions inside the bag and an ad hoc external pressure, which is introduced to confine the fermions within the bag. The fermions are described by the Dirac equation for massless fermions:

\begin{equation*}
    i\gamma^\mu\partial_\mu\psi = 0\quad ,
\end{equation*}
where $\psi$ is the fermion field, and $\gamma^\mu$ are the Dirac matrices. The solution to the Dirac equation is given in terms of the spherical Bessel functions of the zeroth and first order, $j_0(p_0r)$ and $j_1(p_0r)$, where $p_0$ is the energy of the fermion:

\begin{equation*}
    \psi = \mathcal{N} e^{-ip_0t} \begin{pmatrix} j_0(p_0r)\chi^+ \\ \vec{\sigma}\cdot\hat{r}j_1(p_0r)\chi^-\end{pmatrix}\quad ,
\end{equation*}
where $\chi^+$ and $\chi^-$ are the two components of the fermion four dimentional spinor $\psi$, and $\vec{\sigma}$ are the Pauli matrices. The colour flux at a point $r$ inside the bag is given by:

\begin{equation*}
    j_{ab}^\mu(r) = \bar{\psi_a}(r)\gamma^\mu\psi_b(r)\quad ,
\end{equation*}
where $a$ and $b$ are the colour indices of the fermions. If the quantum numbers are not to be lost through the surface of the bag, which is the definition of confinement, then:

\begin{equation*}
  n_\mu j_{ab}^\mu(r) = \bar{\psi_a}(r)\gamma\cdot n \psi_b(r) = 0\quad ,
\end{equation*}
on the surface, where $n$ is a unit space-like vector normal to the surface. Using the gamma properties, $(i\gamma\cdot n)^2 = 1$, so that by assuming that $i\gamma\cdot n = + 1$, the boundary condition on the surface of the bag is given by:

\begin{equation*}
    \bar{\psi}(R)\psi(R) = 0\quad ,
\end{equation*}
leading to the solution of the Dirac equation in the bag:

\begin{equation*}
    \left[j_0\left(p_0R\right)\right]^2 - \left[j_1\left(p_0R\right)\right]^2 = 0\quad ,
\end{equation*}
with solution $p_0R = 2.04$. The total energy inside the bag is given by:

\begin{equation*}
    E = \frac{2.04 N}{R}(\hbar c) + \frac{4\pi}{3}R^3B\quad ,
\end{equation*}
where the first term is the kinetic energy of the fermions, and the second term is the energy due to the presence of an external pressure $B$ which keeps the fermions confined in the bag. The bag pressure is a phenomenological parameter of the model, and it is introduced to confine the fermions within the bag. It can be extracted by minimising the energy of the system with respect to the bag radius $R$, yielding $B=234$ MeV/fm$^3$, for a baryon with $R=0.8$ fm.

\subsection{Deconfinement}
The concept of deconfinement refers to the transition from a confined state to a state where quarks and gluons are no longer confined within hadrons, but are free to move in a larger volume. As modelled by the MIT bag model, non-perturbative QCD effects can be described in terms of an external pressure, which confines quarks and gluons within a finite volume. If the external pressure is overcome by the pressure due to the kinematic energy of the quarks and gluons, then the hadrons constituents are no longer confined, and a transition to a state called Quark-Gluon Plasma (QGP) occurs. The QGP is characterised by extremely high temperatures and energy densities, and it is believed to have existed in the early universe, a few microseconds after the Big Bang. The internal pressure of the bag can increase in two different regimes: by increasing the temperature of the system (hot QGP), or by increasing the baryonic density (cold QGP). For the former, temperatures of about 155 MeV are required to overcome the bag pressure, while for the latter, baryonic densities of about 5 times those of nuclei are needed. The QGP is a unique state of matter that can be studied in the laboratory by colliding heavy ions at high energies. In these collisions, a hot and dense medium is created, where quarks and gluons are no longer confined within hadrons, but are free to move in a larger volume. The study of the QGP is an active area of research in nuclear and particle physics, and it provides valuable insights into the properties of the strong interaction at high energy scales.
Direct observation of the primordial QGP (i.e. that created just after the Big Bang) would provide a wealth of information on the early universe; however, the universe underwent a phase in which electrons were not bound to nuclei, making the universe opaque to electromagnetic radiation, and denying us the possibility of directly observing the QGP. Once the universe cooled enough (3000 K) to allow electrons to bind to nuclei, the electromagnetic radiation decoupled with a black body spectrum of around 3000 K. Since then, as the universe expanded, this electromagnetic radiation has redshifted to a temperature of around 2.7 K, and is denoted as the Cosmic Microwave Background (CBM). The CMB is the oldest light in the universe and provides a snapshot of the universe when it was 300,000 years old, way after the QGP had already cooled down. Hence, the only way to study the QGP is by recreating it in the laboratory, by colliding heavy ions at high energies. In the past decades, several experiments~\cite{ALICE:2022wpn, NA38:2000wlp, NA50:1997hlx, Nouicer:2009fy} have been carried out to study the properties of this state of matter, and the results have provided valuable insights into the properties of the strong interaction at high energy scales.

\subsection{Lattice QCD}
Lattice QCD is a numerical technique used to study the non-perturbative regime of QCD. The method is based on the discretisation of spacetime on a four-dimensional lattice, and the evaluation of the path integral of the theory by Monte Carlo methods, i.e. by sampling possible configurations of the quark and gluon fields according to the probability distribution given by the QCD Lagrangian. The lattice spacing is a parameter of the method, and allows one to avoid the ultraviolet divergences of the theory, which are typical in perturbative QCD, by introducing a cutoff on the momenta of the quark and gluon fields. 
\begin{figure}[htb]
  \centering
  \includegraphics[width=0.7\linewidth]{Figures/Chapter 1/PathIntegrals.png}
  \caption{Feynman introduction to path integrals. Here, a particle emitted from a source at $x_a$ is detected at $x_b$. A finite number of screens, each with a finite number of holes, is placed between the source and the detector. The probability amplitude for the particle to hit the detector is given by the sum of the probabilities of moving from the source to the detector through all possible paths. By adding an infinite amount of screens with an infinite number of holes, and by also considering the time at which the particle passes through the screens, the sum becomes an integral over all possible paths, called a \emph{path integral}.}
  \label{fig:PathIntegrals}
\end{figure}
The Lattice QCD simulations are based on the path integral formalism of quantum field theory~\cite{RevModPhys.20.367}, developed by R. Feynman in the 1940s. The path integral provides a natural extension of the least action principle of classical mechanics to quantum mechanics, and it allows one to calculate the probability amplitude of a particle to move from one point to another in spacetime, considering the evolution of the system over all possible paths. The transition amplitude from the state $(x_a,t_a)$ to the state $(x_b,t_b)$ is given by:
\begin{equation}\label{eq:ampl_prob}
  A \left[(x_a,t_a) \rightarrow (x_b,t_b) \right] = \braket{x_b,t_b | e^{-iH(t_b-t_a)} | x_a,t_a} = \sum_\mathrm{paths} e^{iS[x(t)]}\quad , 
\end{equation}
where $H$ is the Hamiltonian of the system, $S[x(t)]$ is the action of the system, and the sum is over all possible paths from $(x_a,t_a)$ to $(x_b,t_b)$. By taking the continuum limit on space-time, we obtain an integration over all the possible space-time paths of the system:
\begin{equation}\label{eq:path_integral}
  \sum_\mathrm{paths} e^{iS[x(t)]} \rightarrow \int_{x_a}^{x_b} \left[\mathcal{D}x(t)\right] e^{iS[x(t)]}\quad ,
\end{equation}
where the right-hand side term is a functional integral over all possible paths of the system. It is interesting to note that by combining Eqs.~\ref{eq:ampl_prob} and \ref{eq:path_integral}, one gets a quantity resembling the partition function of a statistical system:
\begin{equation*}
  \mathcal{Z} = \sum_{x_a} \braket{x_a,t_a | e^{\beta H} | x_a,t_a}\quad .
\end{equation*}
It is possible to express the partition function in terms of a path integral by applying a Wick rotation to the time variable, $t \rightarrow -i\tau$, with $\tau_a=0\leq\tau\leq\tau_b=\beta$ and considering the Euclidean action in place of the Minkowskian one, $S_E = iS$. Furthermore, since the state at $\tau_a$ is the same as the one at $\tau_b$, a periodic boundary condition is imposed: $x(\tau_a) = x(\tau_b)$. With these considerations, the partition function can be expressed as:
\begin{equation*}
  \mathcal{Z} = \int \left[\mathcal{D}x(\tau)\right] e^{-S_E[x(\tau)]}\quad .
\end{equation*}
This formalism, which was here developed for a single particle, can be extended to a quantum field theory, and in particular to QCD.

The lattice QCD simulations are computationally intensive, and they require large supercomputers to perform the calculations. To limit the computational costs, calculations are often performed at larger up and down quark masses than in nature, drastically reducing the number of virtual quark-antiquark loops that have to be taken into account. Because of the employed Monte Carlo approach, only a finite number of configurations can be considered, leading to statistical uncertainties in the lattice QCD results. In order to obtain physical results, several limits have to be taken: i. the continuum limit, i.e. the extrapolation of the lattice spacing to zero, ii. the infinite-volume limit, i.e. the extrapolation of the lattice size to infinity, and iii. the physical quark-mass limit, i.e. the extrapolation to physical quark masses, although many present-day lattice calculations are already performed directly at, or very close to, the physical values of the quark masses, so that the latter extrapolation becomes less of an issue. 

The results of the lattice QCD simulations are in good agreement with the experimental data, and they have provided valuable insights into the properties of the strong interaction at low energy scales. For example, Fig.~\ref{fig:LQCD_hadron_mass} shows the spectrum of hadrons obtained from lattice QCD simulations, taken from~\cite{BMW:2008jgk}, compared to the experimental data. The agreement between the lattice QCD results and the experimental data is remarkable, confirming the validity of the QCD framework at low energy scales.
\begin{figure}
  \centering
  \includegraphics[width=0.7\linewidth]{Figures/Chapter 1/LQCD_hadron_mass.png}
  \caption{The light hadron spectrum of QCD. Horizontal lines and bands are the experimental values with their decay widths. Lattice QCD results~\cite{BMW:2008jgk} are shown by solid circles. Vertical error bars represent the combined statistical and systematic error estimates. $\pi$, K and $\Xi$ have no error bars, because they are used to set the light quark mass, the strange quark mass and the overall scale, respectively.}
  \label{fig:LQCD_hadron_mass}
\end{figure}